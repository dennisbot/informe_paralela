
%% bare_conf.tex
%% V1.3
%% 2007/01/11
%% by Michael Shell
%% See:
%% http://www.michaelshell.org/
%% for current contact information.
%%
%% This is a skeleton file demonstrating the use of IEEEtran.cls
%% (requires IEEEtran.cls version 1.7 or later) with an IEEE conference paper.
%%
%% Support sites:
%% http://www.michaelshell.org/tex/ieeetran/
%% http://www.ctan.org/tex-archive/macros/latex/contrib/IEEEtran/
%% and
%% http://www.ieee.org/

%%*************************************************************************
%% Legal Notice:
%% This code is offered as-is without any warranty either expressed or
%% implied; without even the implied warranty of MERCHANTABILITY or
%% FITNESS FOR A PARTICULAR PURPOSE!
%% User assumes all risk.
%% In no event shall IEEE or any contributor to this code be liable for
%% any damages or losses, including, but not limited to, incidental,
%% consequential, or any other damages, resulting from the use or misuse
%% of any information contained here.
%%
%% All comments are the opinions of their respective authors and are not
%% necessarily endorsed by the IEEE.
%%
%% This work is distributed under the LaTeX Project Public License (LPPL)
%% ( http://www.latex-project.org/ ) version 1.3, and may be freely used,
%% distributed and modified. A copy of the LPPL, version 1.3, is included
%% in the base LaTeX documentation of all distributions of LaTeX released
%% 2003/12/01 or later.
%% Retain all contribution notices and credits.
%% ** Modified files should be clearly indicated as such, including  **
%% ** renaming them and changing author support contact information. **
%%
%% File list of work: IEEEtran.cls, IEEEtran_HOWTO.pdf, bare_adv.tex,
%%                    bare_conf.tex, bare_jrnl.tex, bare_jrnl_compsoc.tex
%%*************************************************************************

% *** Authors should verify (and, if needed, correct) their LaTeX system  ***
% *** with the testflow diagnostic prior to trusting their LaTeX platform ***
% *** with production work. IEEE's font choices can trigger bugs that do  ***
% *** not appear when using other class files.                            ***
% The testflow support page is at:
% http://www.michaelshell.org/tex/testflow/



% Note that the a4paper option is mainly intended so that authors in
% countries using A4 can easily print to A4 and see how their papers will
% look in print - the typesetting of the document will not typically be
% affected with changes in paper size (but the bottom and side margins will).
% Use the testflow package mentioned above to verify correct handling of
% both paper sizes by the user's LaTeX system.
%
% Also note that the "draftcls" or "draftclsnofoot", not "draft", option
% should be used if it is desired that the figures are to be displayed in
% draft mode.
%
\documentclass[conference]{IEEEtran}
\usepackage[utf8]{inputenc}
\usepackage{blindtext}
\usepackage{amssymb, amsmath}
\usepackage[dvips]{graphicx}
\usepackage{epsfig}
\usepackage{xcolor}
\usepackage{color}
\usepackage{rotating}
\usepackage{textcomp}
\usepackage{ifthen}
\usepackage{comment}
%\usepackage{algorithm}
%\usepackage{algorithmic}
\usepackage[pdfborder={0 0 0}]{hyperref}
\usepackage{verbatim}
\usepackage{listings}
\usepackage{wrapfig}
\usepackage{cleveref}
\usepackage{hyperref}
\let\labelindent\relax
\usepackage{enumitem}
\usepackage{kantlipsum}
\usepackage{array}
\lstnewenvironment{code}{\lstset{language=C,basicstyle=\scriptsize\ttfamily}}{}

\definecolor{bloodred}{rgb}{0.7,0.0,0.0}
\newcommand{\todo}[1]{%
  \noindent{%
    \color{bloodred}{%
      \textbf{TODO:} #1%
    }%
  }%
}

\newcommand{\note}[1]{%
  \noindent{%
    \color{blue}{%
      \textbf{NOTE:} #1%
    }%
  }%
}

\hyphenation{data-type}

\lstdefinestyle{customc}{
  belowcaptionskip=1\baselineskip,
  breaklines=true,
  frame=L,
  xleftmargin=\parindent,
  language=C,
  showstringspaces=false,
  basicstyle=\footnotesize\ttfamily,
  keywordstyle=\bfseries\color{green!40!black},
  commentstyle=\itshape\color{purple!40!black},
  identifierstyle=\color{blue},
  stringstyle=\color{orange},
}
\lstset{escapechar=@,style=customc}
\renewcommand{\lstlistingname}{Código fuente}
%\usepackage[width=7.25in,height=9.53in]{geometry}
% Add the compsoc option for Computer Society conferences.
%
% If IEEEtran.cls has not been installed into the LaTeX system files,
% manually specify the path to it like:
% \documentclass[conference]{../sty/IEEEtran}





% Some very useful LaTeX packages include:
% (uncomment the ones you want to load)


% *** MISC UTILITY PACKAGES ***
%
%\usepackage{ifpdf}
% Heiko Oberdiek's ifpdf.sty is very useful if you need conditional
% compilation based on whether the output is pdf or dvi.
% usage:
% \ifpdf
%   % pdf code
% \else
%   % dvi code
% \fi
% The latest version of ifpdf.sty can be obtained from:
% http://www.ctan.org/tex-archive/macros/latex/contrib/oberdiek/
% Also, note that IEEEtran.cls V1.7 and later provides a builtin
% \ifCLASSINFOpdf conditional that works the same way.
% When switching from latex to pdflatex and vice-versa, the compiler may
% have to be run twice to clear warning/error messages.






% *** CITATION PACKAGES ***
%
%\usepackage{cite}
% cite.sty was written by Donald Arseneau
% V1.6 and later of IEEEtran pre-defines the format of the cite.sty package
% \cite{} output to follow that of IEEE. Loading the cite package will
% result in citation numbers being automatically sorted and properly
% "compressed/ranged". e.g., [1], [9], [2], [7], [5], [6] without using
% cite.sty will become [1], [2], [5]--[7], [9] using cite.sty. cite.sty's
% \cite will automatically add leading space, if needed. Use cite.sty's
% noadjust option (cite.sty V3.8 and later) if you want to turn this off.
% cite.sty is already installed on most LaTeX systems. Be sure and use
% version 4.0 (2003-05-27) and later if using hyperref.sty. cite.sty does
% not currently provide for hyperlinked citations.
% The latest version can be obtained at:
% http://www.ctan.org/tex-archive/macros/latex/contrib/cite/
% The documentation is contained in the cite.sty file itself.






% *** GRAPHICS RELATED PACKAGES ***
%
\ifCLASSINFOpdf
  % \usepackage[pdftex]{graphicx}
  % declare the path(s) where your graphic files are
  % \graphicspath{{../pdf/}{../jpeg/}}
  % and their extensions so you won't have to specify these with
  % every instance of \includegraphics
  % \DeclareGraphicsExtensions{.pdf,.jpeg,.png}
\else
  % or other class option (dvipsone, dvipdf, if not using dvips). graphicx
  % will default to the driver specified in the system graphics.cfg if no
  % driver is specified.
  % \usepackage[dvips]{graphicx}
  % declare the path(s) where your graphic files are
  % \graphicspath{{../eps/}}
  % and their extensions so you won't have to specify these with
  % every instance of \includegraphics
  % \DeclareGraphicsExtensions{.eps}
\fi
% graphicx was written by David Carlisle and Sebastian Rahtz. It is
% required if you want graphics, photos, etc. graphicx.sty is already
% installed on most LaTeX systems. The latest version and documentation can
% be obtained at:
% http://www.ctan.org/tex-archive/macros/latex/required/graphics/
% Another good source of documentation is "Using Imported Graphics in
% LaTeX2e" by Keith Reckdahl which can be found as epslatex.ps or
% epslatex.pdf at: http://www.ctan.org/tex-archive/info/
%
% latex, and pdflatex in dvi mode, support graphics in encapsulated
% postscript (.eps) format. pdflatex in pdf mode supports graphics
% in .pdf, .jpeg, .png and .mps (metapost) formats. Users should ensure
% that all non-photo figures use a vector format (.eps, .pdf, .mps) and
% not a bitmapped formats (.jpeg, .png). IEEE frowns on bitmapped formats
% which can result in "jaggedy"/blurry rendering of lines and letters as
% well as large increases in file sizes.
%
% You can find documentation about the pdfTeX application at:
% http://www.tug.org/applications/pdftex





% *** MATH PACKAGES ***
%
%\usepackage[cmex10]{amsmath}
% A popular package from the American Mathematical Society that provides
% many useful and powerful commands for dealing with mathematics. If using
% it, be sure to load this package with the cmex10 option to ensure that
% only type 1 fonts will utilized at all point sizes. Without this option,
% it is possible that some math symbols, particularly those within
% footnotes, will be rendered in bitmap form which will result in a
% document that can not be IEEE Xplore compliant!
%
% Also, note that the amsmath package sets \interdisplaylinepenalty to 10000
% thus preventing page breaks from occurring within multiline equations. Use:
%\interdisplaylinepenalty=2500
% after loading amsmath to restore such page breaks as IEEEtran.cls normally
% does. amsmath.sty is already installed on most LaTeX systems. The latest
% version and documentation can be obtained at:
% http://www.ctan.org/tex-archive/macros/latex/required/amslatex/math/





% *** SPECIALIZED LIST PACKAGES ***
%
%\usepackage{algorithmic}
% algorithmic.sty was written by Peter Williams and Rogerio Brito.
% This package provides an algorithmic environment fo describing algorithms.
% You can use the algorithmic environment in-text or within a figure
% environment to provide for a floating algorithm. Do NOT use the algorithm
% floating environment provided by algorithm.sty (by the same authors) or
% algorithm2e.sty (by Christophe Fiorio) as IEEE does not use dedicated
% algorithm float types and packages that provide these will not provide
% correct IEEE style captions. The latest version and documentation of
% algorithmic.sty can be obtained at:
% http://www.ctan.org/tex-archive/macros/latex/contrib/algorithms/
% There is also a support site at:
% http://algorithms.berlios.de/index.html
% Also of interest may be the (relatively newer and more customizable)
% algorithmicx.sty package by Szasz Janos:
% http://www.ctan.org/tex-archive/macros/latex/contrib/algorithmicx/




% *** ALIGNMENT PACKAGES ***
%
%\usepackage{array}
% Frank Mittelbach's and David Carlisle's array.sty patches and improves
% the standard LaTeX2e array and tabular environments to provide better
% appearance and additional user controls. As the default LaTeX2e table
% generation code is lacking to the point of almost being broken with
% respect to the quality of the end results, all users are strongly
% advised to use an enhanced (at the very least that provided by array.sty)
% set of table tools. array.sty is already installed on most systems. The
% latest version and documentation can be obtained at:
% http://www.ctan.org/tex-archive/macros/latex/required/tools/


%\usepackage{mdwmath}
%\usepackage{mdwtab}
% Also highly recommended is Mark Wooding's extremely powerful MDW tools,
% especially mdwmath.sty and mdwtab.sty which are used to format equations
% and tables, respectively. The MDWtools set is already installed on most
% LaTeX systems. The lastest version and documentation is available at:
% http://www.ctan.org/tex-archive/macros/latex/contrib/mdwtools/


% IEEEtran contains the IEEEeqnarray family of commands that can be used to
% generate multiline equations as well as matrices, tables, etc., of high
% quality.


%\usepackage{eqparbox}
% Also of notable interest is Scott Pakin's eqparbox package for creating
% (automatically sized) equal width boxes - aka "natural width parboxes".
% Available at:
% http://www.ctan.org/tex-archive/macros/latex/contrib/eqparbox/





% *** SUBFIGURE PACKAGES ***
%\usepackage[tight,footnotesize]{subfigure}
% subfigure.sty was written by Steven Douglas Cochran. This package makes it
% easy to put subfigures in your figures. e.g., "Figure 1a and 1b". For IEEE
% work, it is a good idea to load it with the tight package option to reduce
% the amount of white space around the subfigures. subfigure.sty is already
% installed on most LaTeX systems. The latest version and documentation can
% be obtained at:
% http://www.ctan.org/tex-archive/obsolete/macros/latex/contrib/subfigure/
% subfigure.sty has been superceeded by subfig.sty.



%\usepackage[caption=false]{caption}
%\usepackage[font=footnotesize]{subfig}
% subfig.sty, also written by Steven Douglas Cochran, is the modern
% replacement for subfigure.sty. However, subfig.sty requires and
% automatically loads Axel Sommerfeldt's caption.sty which will override
% IEEEtran.cls handling of captions and this will result in nonIEEE style
% figure/table captions. To prevent this problem, be sure and preload
% caption.sty with its "caption=false" package option. This is will preserve
% IEEEtran.cls handing of captions. Version 1.3 (2005/06/28) and later
% (recommended due to many improvements over 1.2) of subfig.sty supports
% the caption=false option directly:
%\usepackage[caption=false,font=footnotesize]{subfig}
%
% The latest version and documentation can be obtained at:
% http://www.ctan.org/tex-archive/macros/latex/contrib/subfig/
% The latest version and documentation of caption.sty can be obtained at:
% http://www.ctan.org/tex-archive/macros/latex/contrib/caption/




% *** FLOAT PACKAGES ***
%
%\usepackage{fixltx2e}
% fixltx2e, the successor to the earlier fix2col.sty, was written by
% Frank Mittelbach and David Carlisle. This package corrects a few problems
% in the LaTeX2e kernel, the most notable of which is that in current
% LaTeX2e releases, the ordering of single and double column floats is not
% guaranteed to be preserved. Thus, an unpatched LaTeX2e can allow a
% single column figure to be placed prior to an earlier double column
% figure. The latest version and documentation can be found at:
% http://www.ctan.org/tex-archive/macros/latex/base/



%\usepackage{stfloats}
% stfloats.sty was written by Sigitas Tolusis. This package gives LaTeX2e
% the ability to do double column floats at the bottom of the page as well
% as the top. (e.g., "\begin{figure*}[!b]" is not normally possible in
% LaTeX2e). It also provides a command:
%\fnbelowfloat
% to enable the placement of footnotes below bottom floats (the standard
% LaTeX2e kernel puts them above bottom floats). This is an invasive package
% which rewrites many portions of the LaTeX2e float routines. It may not work
% with other packages that modify the LaTeX2e float routines. The latest
% version and documentation can be obtained at:
% http://www.ctan.org/tex-archive/macros/latex/contrib/sttools/
% Documentation is contained in the stfloats.sty comments as well as in the
% presfull.pdf file. Do not use the stfloats baselinefloat ability as IEEE
% does not allow \baselineskip to stretch. Authors submitting work to the
% IEEE should note that IEEE rarely uses double column equations and
% that authors should try to avoid such use. Do not be tempted to use the
% cuted.sty or midfloat.sty packages (also by Sigitas Tolusis) as IEEE does
% not format its papers in such ways.





% *** PDF, URL AND HYPERLINK PACKAGES ***
%
%\usepackage{url}
% url.sty was written by Donald Arseneau. It provides better support for
% handling and breaking URLs. url.sty is already installed on most LaTeX
% systems. The latest version can be obtained at:
% http://www.ctan.org/tex-archive/macros/latex/contrib/misc/
% Read the url.sty source comments for usage information. Basically,
% \url{my_url_here}.





% *** Do not adjust lengths that control margins, column widths, etc. ***
% *** Do not use packages that alter fonts (such as pslatex).         ***
% There should be no need to do such things with IEEEtran.cls V1.6 and later.
% (Unless specifically asked to do so by the journal or conference you plan
% to submit to, of course. )


% correct bad hyphenation here
\hyphenation{op-tical net-works semi-conduc-tor}
\hypersetup{colorlinks=false,linkbordercolor=red,linkcolor=green,pdfborderstyle={/S/U/W 1}}

\begin{document}
%
% paper title
% can use linebreaks \\ within to get better formatting as desired
\title{Explorando el soporte para large-count en MPI-3\\¡Al \texttt{INT\_MAX} y más allá!}


% author names and affiliations
% use a multiple column layout for up to three different
% affiliations
\author{\IEEEauthorblockN{Pavel Mendoza}
\IEEEauthorblockA{Maestría en Ciencias\\de la Computación\\
Universidad Católica San Pablo\\
Email: pmendozav@gmail.com}
\and
\IEEEauthorblockN{Dennis Huillca}
\IEEEauthorblockA{Maestría en Ciencias\\de la Computación\\
Universidad Católica San Pablo\\
Email: dennisbot@gmail.com}
}

% conference papers do not typically use \thanks and this command
% is locked out in conference mode. If really needed, such as for
% the acknowledgment of grants, issue a \IEEEoverridecommandlockouts
% after \documentclass

% for over three affiliations, or if they all won't fit within the width
% of the page, use this alternative format:
%
%\author{\IEEEauthorblockN{Michael Shell\IEEEauthorrefmark{1},
%Homer Simpson\IEEEauthorrefmark{2},
%James Kirk\IEEEauthorrefmark{3},
%Montgomery Scott\IEEEauthorrefmark{3} and
%Eldon Tyrell\IEEEauthorrefmark{4}}
%\IEEEauthorblockA{\IEEEauthorrefmark{1}School of Electrical and Computer Engineering\\
%Georgia Institute of Technology,
%Atlanta, Georgia 30332--0250\\ Email: see http://www.michaelshell.org/contact.html}
%\IEEEauthorblockA{\IEEEauthorrefmark{2}Twentieth Century Fox, Springfield, USA\\
%Email: homer@thesimpsons.com}
%\IEEEauthorblockA{\IEEEauthorrefmark{3}Starfleet Academy, San Francisco, California 96678-2391\\
%Telephone: (800) 555--1212, Fax: (888) 555--1212}
%\IEEEauthorblockA{\IEEEauthorrefmark{4}Tyrell Inc., 123 Replicant Street, Los Angeles, California 90210--4321}}




% use for special paper notices
%\IEEEspecialpapernotice{(Invited Paper)}




% make the title area
\maketitle


\begin{abstract}
%\boldmath
Para describir una región de memoria en MPI, el estándar usa el par
(count, datatype). La especificación de C para esta convención usa
un tipo \texttt{int} para el count. Dado que los tipos \texttt{int} de C son
casi siempre 32 bits de largo y con signo, el contar más de
$2^{31}$ elementos resulta un reto. En lugar de cambiar
las rutinas existentes de MPI y en consecuencia a todos los
consumidores de estas rutinas, en los foros de MPI se asegura
que los usuarios pueden crear grandes tipos de datos a partir de tipos más pequeños.
Para evaluar esta hipótesis y para proveer una solución user-friendly para el
problema del large-count, fue desarrollada la librería BigMPI~\cite{large-count-problem},
una librería construida encima de MPI que mapea funciones
large-count parecidas a las de MPI a aquellas de MPI-3 con sus características
estándar. BigMPI demuestra una forma de realizar dicha construcción y
revela fallos en el actual estándar MPI.
En el presente articulo pretendemos demostrar
su desempeño y el grado de overhead que añade su uso a través de la
prueba en un cluster de 3 computadores Xeon E3-1200 4th Gen de 4 Núcleos.
\end{abstract}
% IEEEtran.cls defaults to using nonbold math in the Abstract.
% This preserves the distinction between vectors and scalars. However,
% if the journal you are submitting to favors bold math in the abstract,
% then you can use LaTeX's standard command \boldmath at the very start
% of the abstract to achieve this. Many IEEE journals frown on math
% in the abstract anyway.

% Note that keywords are not normally used for peerreview papers.
\begin{IEEEkeywords}
BigMPI, \texttt{MPI\_Count}, \texttt{MPI\_Aint}, \texttt{size\_t}, ILP32, IL32P64, I32LP64.
\end{IEEEkeywords}



% For peer review papers, you can put extra information on the cover
% page as needed:
% \ifCLASSOPTIONpeerreview
% \begin{center} \bfseries EDICS Category: 3-BBND \end{center}
% \fi
%
% For peerreview papers, this IEEEtran command inserts a page break and
% creates the second title. It will be ignored for other modes.
\IEEEpeerreviewmaketitle



\section{Introducción}
MPI (Message Passing Interface)~\cite{{mpiforum:94, mpiforum:96, mpiforum:09, mpiforum:12}},
define un amplio conjunto de funcionalidades
para escribir programas paralelos, especialmente entre sistemas de computación
distribuida. Ahora, después de más de 22 años, MPI continúa siendo ampliamente
usado y ha llegado incluso a ser usado en un millón de
núcleos~\cite{{balaji2011mpi}}, para poder
escalar en términos del tamaño del problema, uno necesita describir grandes
conjuntos de datos, la existencia del par (count, datatype) funciona bien hasta
que "count" excede el rango del tipo entero nativo (en el caso de la interface C
el tipo es \texttt{int}, el cuál es de 32 bits en la mayoría de plataformas).
A esto le llamamos el problema del "large-count".

Cuando se realizó el borrador de MPI-3, el foro de MPI tomó poca importancia
al soporte de "large-count"~\cite{ticket265}. El foro introdujo un conjunto de rutinas MPI\_Foo\_x
que proveían un equivalente al "large-count" de un MPI\_Foo logrando así al menos
tener un soporte rudimentario al problema del "large-count". Para ser explícito
en este contexto, Foo es "Get\_elements", "Type\_size", "Type\_get\_extend",
"Type\_get\_true\_extend" y "Status\_set\_elements", el cuál es el conjunto mínimo de
funciones que debe soportar el problema del "large-count" para poder manipular
tipos de datos derivados que representan "large-counts". Después de largas
deliberaciones, el foro concluyó: "sólo usen tipos de datos" es la solución
suficiente para los usuarios~\cite{squyres-blog-large-count}. Por ejemplo uno puede describir 4 mil millones de bytes
como mil millones de enteros de 4 bytes. O, uno puede usar tipos de datos MPI
contiguos para describir 16 mil millones de bytes como 1,000 16 millones de bloques.
Para estos ejemplos simples, uno puede fácilmente visualizar una solución. Sólo a
través de la implementación al enfoque propuesto para todos los casos en MPI uno
descubre los retos escondidos en tales aseveraciones.

BigMPI provee un librería al más alto nivel que se esfuerza en soportar "large-counts".
Fue escrito para probar la aseveración del foro en el que los tipos de datos son
suficientes para el soporte del "large-count" y como para crear una librería como solución
para aplicaciones que requerían soporte para el "large-count". En este contexto, "large-count"
es cualquier cantidad que excede \texttt{INT\_MAX}. BigMPI hace el menor cambio posible a las
rutinas estándar de MPI para habilitar "large-counts", minimizando los cambios en la aplicación.

BigMPI está diseñado para el caso común donde uno tiene un espacio de direcciones de 64 bits
y no puede hacer comunicaciones MPI en más de $2^{31}$ elementos a pesar de tener suficiente
memoria para reservar tales buffers. Dado que los sistemas con más de $2^{63}$ bytes (8192 PiB)
de memoria por nodo son improbables que existan en un futuro cercano---la capacidad total del
sistema de memoria para una máquina de exa-escala ha sido predicha a ser de 50--100
petabytes~\cite{shalf2011exascale}---
BigMPI no se esfuerza en soportar todo ese rango de \texttt{MPI\_Count}
(posiblemente un entero de 128 bits) internamente; sino más bien
usa \texttt{size\_t} y \texttt{MPI\_Aint} dado que estos tipos
reflejan realmente los límites de la memoria disponible en lugar del tamaño teórico del
archivo del sistema (tal como \texttt{MPI\_Count} lo hace).

\section{Antecedentes}
El estándar MPI provee un amplio rango de funciones de comunicación que toman como argumento
un entero \texttt{int} en C para la cantidad de elementos a transmitir, de este modo limitando
este valor a \texttt{INT\_MAX} o menos, por lo tanto, uno no puede enviar por ejemplo, 3 mil
millones de bytes usando el tipo de datos \texttt{MPI\_BYTE} o un vector de 5 mil millones de
enteros usando el tipo \texttt{MPI\_INT}.

Estas limitaciones podrían parecer académicas: 2 mil millones de \texttt{MPI\_DOUBLE} es igual
a 16 GB, y uno podría pensar que las aplicaciones podrían raramente necesitar transmitir tal
cantidad de datos, dado que podría haber menos de la memoria disponible para todo el espacio
de direcciones en el cuál el proceso MPI está ejecutándose. Dos tendencias recientes podrían
mostrar esta limitación incrementalmente no práctica, sin embargo: primero, el crecimiento de
poder computacional por nodo implica el incremento de datos por proceso MPI dentro de un contexto
de escalamiento débil, y segundo, aplicaciones de Big Data podría requerir más memoria por proceso
que los códigos de simulación tradicionales que resuelven la ecuación de movimiento para un
dominio particular de las ciencias físicas.

\begin{figure}
\begin{code}
int MPIX_Send_x(const void *buf, MPI_Count count,
                MPI_Datatype dt, int dest,
                int tag, MPI_Comm comm)
{
    int rc = MPI_SUCCESS;
    if (likely (count <= INT_MAX )) {
        rc = MPI_Send(buf, (int)count, dt, dest, tag, comm);
    } else {
        MPI_Datatype newtype;
        MPIX_Type_contiguous_x(count, dt, &newtype);
        MPI_Type_commit(&newtype);
        rc = MPI_Send(buf, 1, newtype, dest, tag, comm);
        MPI_Type_free(&newtype);
    }
    return rc;
}
\end{code}
\caption{Implementación de un envío usando un "large-count", el cuál sirve como plantilla
para muchas otras rutinas MPI-3.\label{code:mpi_send_x}}
\end{figure}

En la publicación de referencia usada para este articulo se hace referencia
también a los problemas con las interfaces C, y usamos las bien conocidas convenciones
I$n_{I}$L$n_{L}$P$n_{P}$ para referirnos a los tamaños de los tipos C \texttt{int}, \texttt{long}
, \texttt{void*}, respectivamente. Para sistemas ILP32, el buffer más grande que uno puede
reservar es $2^{32}$ bytes (4 GiB), mientras que MPI puede manipular buffers de hasta 2 GiB;
el factor de diferencia 2 casi nunca es un problema dado que 4 GiB de \texttt{int}, por ejemplo
requiere un count de sólo $2^{30}$. El problema surge en sistemas de IL32P64 y I32LP64 porque
uno puede reservar más memoria en un buffer del que puede ser capturado con un count entero y
un tipo de dato pre-incluido, por ejemplo un vector de 3 mil millones de floats requiere
12 GB de memoria pero no puede ser comunicado con ninguna rutina de comunicación usando
tipos de datos pre-incluidos.

\section{Diseño principal}
En esta sección se describe el mapeo de una variante de funciones de comunicación al
estilo MPI hacia funciones MPI-3. Esta tarea usualmente involucra crear un tipo de datos
"large-count". BigMPI implementa todas las variantes de (bcast, gather, scatter, allgather, alltoall
) y RMA (put, get, accumulate, get\_accumulate) por ejemplo la función Send en la Figura 1.
Esta clase de rutinas provee la funcionalidad MPI comunmente usada.

Las función crítica en todas las implementaciones de "large-count" vistas arriba es
\texttt{MPIX\_Type\_contiguous\_x}, el cual emite un único tipo de dato que representa hasta \texttt{SIZE\_MAX}
elementos. Esta rutina de utilidad nos permite implementar soporte al "large-count" de una
forma sencilla dado que todas las instancias de $(large\_count,type)$ son mapeadas a $(1,large\_type)$
por esta función. La Figura 2 muestra tal implementación.

% split MPI_TYPE_CREATE_STRUCT below on two lines to fix an overfull warning
\begin{figure}
\begin{code}
int MPIX_Type_contiguous_x(MPI_Count count,
                           MPI_Datatype oldtype,
                           MPI_Datatype * newtype)
{
    assert(count<SIZE_MAX);
    MPI_Count c = count/INT_MAX, r = count%INT_MAX;

    MPI_Datatype chunks, remainder;
    MPI_Type_vector(c, INT_MAX, INT_MAX, oldtype, &chunks);
    MPI_Type_contiguous(r, oldtype, &remainder);

    MPI_Aint lb , extent;
    MPI_Type_get_extent(oldtype, &lb, &extent);

    MPI_Aint remdisp          = (MPI_Aint)c*INT_MAX*extent;
    int blklens[2]            = {1,1};
    MPI_Aint disps[2]         = {0,remdisp};
    MPI_Datatype types[2]     = {chunks,remainder};
    MPI_Type_create_struct(2, blklens, disps, types,
             newtype);

    MPI_Type_free(&chunks);
    MPI_Type_free(&remainder);

    return MPI_SUCCESS;
}
\end{code}
\caption{Función para la construcción de un tipo de dato contiguo "large-count".
un vector de tipo describe una serie de bloques adyacentes, y un tipo de estructura
recoje cualquier dato restante en caso el count no sea divisible de forma exacta.}
\label{code:type_contig_x}
\end{figure}

\section{Implementación de un Cluster MPI}

En esta sección se describe las instrucciones llevadas a cabo para construir un cluster MPI
el cual se usarán posteriormente para probar el rendimiento de la librería
BigMPI~\cite{big-mpi-library}, para mayor referencia ver~\cite{mpi-cluster-lan-raspberry} y
~\cite{mpi-cluster-lan}.

\subsection{Compilando MPI}
\begin{enumerate}
\item Refrescar la lista de paquetes y sus versiones en caché.
  \begin{itemize}
  \item[--] \texttt{\$ sudo apt-get update}
  \end{itemize}
\item Instala nuevas versiones
  \begin{itemize}
  \item[--] \texttt{\$ sudo apt-get upgrade}
  \end{itemize}
\item hacer un directorio para poner las fuentes ahí
  \begin{itemize}
  \item[--] \texttt{\$ mkdir /home/username/mpich2}
  \item[--] \texttt{\$ cd \texttildelow/mpich2}
  \end{itemize}
\item Obtener el código fuente MPI de Argonne.
  \begin{itemize}
  \item[--] \texttt{\href{http://www.mcs.anl.gov/research/projects/mpich2/downloads/tarballs/1.4.1p1/mpich2-1.4.1p1.tar.gz}{wget URL}}
  \item[--] notar que se puede navegar \href{http://www.mpich.org/downloads/}{aquí} para
obtener la última versión.
  \end{itemize}
\item Desempaquetarlo (usar la versión que corresponda)
  \begin{itemize}
  \item[--] \texttt{\$ tar xfz mpich2-1.4.1p1.tar.gz}
  \end{itemize}
\item Create un lugar para poner tus compilaciones - te ayudará para darte cuenta que de nuevo
pusiste en tu sistema.
  \begin{itemize}
  \item[--] \texttt{\$ sudo mkdir /home/username/}
  \item[--] \texttt{\$ sudo mkdir /home/username/mpich2-install}
  \end{itemize}
\item Create un directorio de compilación (de manera que tendremos el directorio de código limpio)
  \begin{itemize}
  \item[--] \texttt{\$ mkdir /home/username/mpich\_build}
  \end{itemize}
\item Cambiamos al directorio build
  \begin{itemize}
  \item[--] \texttt{\$ cd /home/username/mpich\_build}
  \end{itemize}
\item Ahora configuramos el build
  \begin{itemize}
  \item[--] \texttt{\$ sudo /home/username/mpich2/\\mpich2-1.4.1p1/configure -prefix=/home/username/
mpich2-install}
  \end{itemize}
\item Hacer make a los archivos
  \begin{itemize}
  \item[--] \texttt{\$ sudo make}
  \end{itemize}
\item Instalar los archivos
  \begin{itemize}
  \item[--] \texttt{\$ sudo make install}
  \end{itemize}
\item Añadir la ubicación de tu instalación en el PATH
  \begin{itemize}
  \item[--] \texttt{\$ export PATH=\$PATH:/home/\\username/mpich2-install/bin}
  \end{itemize}
\item Notar que para añadir permanentemente al PATH necesitaremos editar
el archivo \texttt{\texttildelow/.profile} y dentro añadir las sgte línea.
  \begin{itemize}
  \item[--] \texttt{PATH="\$PATH:/home\\/username/mpich2/bin"}
  \end{itemize}
\item Revisamos si las cosas se instalaron o no
  \begin{itemize}
  \item[--] \texttt{\$ which mpicc}
  \item[--] \texttt{\$ which mpiexec}
  \end{itemize}
\item Cambiar el directorio de regreso a home y crear un directorio
para hacer tus pruebas.
  \begin{itemize}
  \item[--] \texttt{\$ cd \texttildelow}
  \item[--] \texttt{\$ mkdir mpi\_testing}
  \item[--] \texttt{\$ cd mpi\_testing}
  \end{itemize}
\item Ahora probamos si MPI funciona en un único nodo
  \begin{itemize}
  \item[--] \texttt{\$ mpiexec -f machinefile -n <number> master}
  \end{itemize}
\end{enumerate}
Donde machinefile contiene una lista de direcciones IP (en este caso)
\begin{enumerate}[resume]
\item Ahora probamos si MPI funciona en un único nodo para las máquinas.
  \begin{itemize}
  \item[--] Obtén tu dirección IP:
  \item[--] \texttt{\$ ifconfig}
  \item[--] Poner esto dentro de un único archivo llamado\\machilefile:
  \item[--] \texttt{nano machinefile}
  \item[--] Add this line:
  \item[--] \texttt{192.168.1.161} [o la IP que tengas]
  \end{itemize}
\end{enumerate}
Ahora ejecutamos código C. En el ejemplo de subdirectorio de donde
compilas MPI.
\begin{enumerate}[resume]
\item \texttt{\$ cd /home/username/mpi\_testing}
\item \texttt{\$ mpiexec -f machinefile -n 2 \texttildelow/mpich\_build/examples/cpi}
  \begin{itemize}
    \item[---] La salida es:
    \begin{itemize}
      \item[--] Proceso 0 de 2 en master
      \item[--] Proceso 1 de 2 en master
    \end{itemize}
  \end{itemize}
\end{enumerate}

\subsection{Usando SSH en lugar de password entre PCs}
En esta sub-sección se muestra como crear una cuenta SSH para poder manipular los
equipos que se usarán como parte del cluster, para mayor referencia revisar
~\cite{how-to-setup-ssh-keys}.
\begin{enumerate}
\item Nos posicionamos en el directorio home de username
  \begin{itemize}
  \item[--] \texttt{\$ cd \texttildelow}
  \end{itemize}
\item Generamos las claves públicas y privadas
  \begin{itemize}
  \item[--] \texttt{\$ ssh-keygen -t rsa "username@master"}
  \end{itemize}
\end{enumerate}
Esto establece la ubicación por defecto de
\texttt{/home/username/.ssh/id\_rsa} para almacenar la clave
insertamos enters (no usaremos passphrase) y nuestras claves estarán
listas para ser usadas.

\begin{enumerate}[resume]
\item Ingresamos nuestra clave pública al archivo de claves autorizadas del cliente
para poder conectarnos sin necesidad de password.
  \begin{itemize}
  \item[--] \texttt{\$ cat \texttildelow/.ssh/id\_rsa.pub | ssh cliente@192.168.1.162 "mkdir .ssh;cat >> .ssh/authorized\_keys"}
  \end{itemize}
\item Si te logueas en la pc cliente y haces:
  \begin{itemize}
  \item[--] \texttt{\$ ls -al \texttildelow/.ssh}
  \item[--] deberías ver un archivo llamada "authorized\_keys" -- este es tu ticket para
no estar usando siempre el login en los nodos al conectarse cada vez usando ssh.
  \end{itemize}
\item Ahora agregemos la nueva máquina del cliente al archivo machinefile
(nos logueamos vía ssh y recuperamos su IP).
  \begin{itemize}
  \item[--] Trabajando en la PC master
  \item[--] \texttt{\$ nano machinefile} y luego añadir:
  \item[] \texttt{192.168.1.161}
  \item[] \texttt{192.168.1.162}
  \item[--] (o según fuera las IPs según su contexto)
  \end{itemize}
\end{enumerate}
\subsection{Instalando NFS}
Ahora tenemos que compartir un directorio vía NFS en la pc \textbf{master} en la
cual los \textbf{clientes} montarán para intercambiar datos.

\textbf{\textit{Servidor NFS}}
\begin{enumerate}
  \item[--] Instalamos los paquetes requeridos con el sgte comando
  \item[] \texttt{\$ sudo apt-get install nfs-kernel-server}
\end{enumerate}
Asumiendo que aún estás logueado como \textbf{username}, vamos a crear un directorio con el
nombre \textbf{cloud} el cual compartiremos en la red.
\begin{enumerate}[resume]
\item[] \texttt{\$ mkdir cloud}
\end{enumerate}
para exportar el directorio \textbf{cloud}, creas una entrada en \textbf{etc/exportfs}
\begin{enumerate}[resume]
\item[] \texttt{\$ cat /etc/exportfs}
\item[] \texttt{\/home/username/cloud *(rw,sync,no\_root\_squash, no\_subtree\_check)}
\end{enumerate}
Aquí, en lugar de \textbf{*} puedes dar a cualquier persona la dirección IP con quién
quieres compartir este directorio, pero esto sólo hará que nuestro trabajo sea más fácil.
\begin{enumerate}[resume]
\item[--] Después de haber hecho una entrada en el paso anterior, ejecutar lo sgte.
\item[] \texttt{exports -a}
\item[--] ejecutar este comando cada vez que hagas un cambio en \texttt{/etc/exports}
\item[--] Si es requerido, reiniciar el servidor \textbf{nfs}
\item[] \texttt{\$ sudo service nfs-kernel-server restart}
\end{enumerate}

\textbf{\textit{Cliente NFS}}

Instalar los paquetes requeridos
\begin{enumerate}
  \item[] \texttt{\$ sudo apt-get install nfs-common}
\end{enumerate}

Crear un directorio en la máquina cliente con el mismo nombre \texttt{cloud}
\begin{enumerate}
  \item[] \texttt{\$ mkdir cloud}
\end{enumerate}

Y ahora, montar el directorio compartido como
\begin{enumerate}
  \item[] \texttt{\$ sudo mount -t nfs master:/home/username/cloud \texttildelow/cloud}
\end{enumerate}

Para verificar los directorios montados
\begin{enumerate}
  \item[] \texttt{\$ df -h}
  \item[] \texttt{Filesystem}
  \item[] \texttt{master:/home/username/cloud}
\end{enumerate}

Para que el montaje sea permanente de forma que no tengas que montar manualmente
el directorio compartido cada vez, tendrás que hacer un reinicio de sistema, puedes
crear una entrada en el sistema de tablas de archivos -i.e el archivo \texttt{/etc/fstab}
como esto.
\begin{enumerate}
  \item[] \texttt{\$ cat /etc/fstab}
  \item[] \texttt{\#MPI CLUSTER SETUP}
  \item[] \texttt{master:/home/username/cloud /home/username/cloud nfs}
\end{enumerate}

Ahora desde la máquina master, copia tus ejecutables en un directorio compartido
\texttt{cloud} o aún mejor, compila tu código dentro del directorio compartido NFS.
\begin{enumerate}
  \item[] \texttt{\$ cd cloud/}
  \item[] \texttt{\$ pwd}
  \item[] \texttt{/home/username/cloud}
\end{enumerate}

Ahora ejecutamos el código C del ejemplo CPI.

Te pedirá llenar tu passphrase (a menos que durante la generación de
las claves pública y privada lo hayas dejado en blanco).
\begin{enumerate}
  \item[] \texttt{\$ cd /home/username/username/cloud}
  \item[] \texttt{\$ mpiexec -f machinefile -n 2 \texttildelow/username/cloud/cpi}
\end{enumerate}

la salida será:

Proceso 0 de 2 en master

Proceso 1 de 2 en master

Si cambias el hostname de tu segunda PC (cliente) y ejecutas:
\begin{enumerate}
  \item[] \texttt{\$ mpiexec -f machinefile -n 2 \texttildelow/username/cloud/cpi}
\end{enumerate}

la salida será:

Proceso 0 de 2 en master

Proceso 1 de 2 en cliente

Felicitaciones, finalmente construiste un cluster de 2 computadoras,
puedes repetir el proceso para muchas más computadoras.

\section{Instalación y uso de la librerías BigMPI}
En esta sección describiremos paso a paso como compilar y hacer uso de la librería BigMPI
dentro del cluster construido previamente.

El proyecto se encuentra en \href{https://github.com/jeffhammond/BigMPI}{github.com/jeffhammond}
podemos realizar una clonación o simplemente descargarlo como zip y luego extraerlo en un directorio
dentro de nuestro sistema de archivos y renombrar la carpeta como por ejemplo.
\begin{enumerate}
\item[] \texttt{\$ cd /home/username/bigmpi}
\end{enumerate}
A partir de aquí empezamos a ejecutar los comandos en el terminal para compilar y ejecutar
los programas que hacen uso de BigMPI.
Creamos un archivo bash para poder ejecutar las siguientes instrucciones que vienen
acompañadas con el paquete BigMPI.
\begin{enumerate}
\item[] \texttt{\$ cmake /home/username/bigmpi -DCMAKE\_C\_COMPILER=mpicc}
\item[] \texttt{\$ export MPI\_IMPL=mpich}
\item[] \texttt{\$ export CC=lang}
\item[] \texttt{\$ sh ./install-deps.sh \$MPI\_IMPL}
\item[] \texttt{\$ sh ./build-run.sh}
\end{enumerate}

Con todas esas instrucciones de línea de comandos podemos a continuación dirigirnos
a la carpeta test en donde se generarán nuestros ejecutables que hacen uso de la librería
BigMPI, a partir de aquí ya podemos experimentar con nuestros programas escritos en MPI.

\section{Pruebas de Rendimiento en un Cluster MPI}
En la presente sub-sección realizaremos varios experimentos dentro del cluster
que acabamos de crear en la subsección anterior, esto con el fin de
probar el rendimiento de la librería BigMPI en un sistema de memoria distribuida.
Para esto usaremos 3 computadoras.

\subsection{Prueba 1 : Ancho de banda del Cluster}
% \kant[1-2]
\begin{table}[htb]
\caption{Transmisión de datos en cluster - Resultados (B/s)}
\label{1234}
\resizebox{\columnwidth}{!}{%
\begin{tabular}{c*{5}{>{$}c<{$}}}
& \text{GigaBytes(GB)}  & \text{MPI (MB / segs.)} & \text{BigMPI (MB / segs.)}\\
& 0.05 & 58.38 & 58.38\\
& 0.1  & 58.38 & 58.38\\
& 0.5  & 58.39 & 58.38\\
& 1    & 58.39 & 58.31\\
& 2    & 58.32 & 58.31\\
& 3    & No Aplicable & 58.31
\end{tabular}
}
\end{table}
% \kant[3-5]
\subsection{Prueba 2 : tiempo de transmisión en el Cluster}
% \kant[1-2]
\begin{table}[htb]
\caption{Transmisión de datos 2GB (s)}
\label{1234}
\resizebox{\columnwidth}{!}{%
\begin{tabular}{c*{5}{>{$}c<{$}}}
& \text{Núcleos(GB)}  & \text{MPI (segs.)} & \text{BigMPI (segs.)}\\
& 2 & 17.15 & 17.14\\
& 4  & 34.7 & 34.7\\
& 6  & 52.27 & 52.25\\
& 8    & 189.6 & 99.31\\
& 10    & 176.6 & 158.8\\
& 12    & 307.2 & 319.3
\end{tabular}
}
\end{table}
% \kant[3-5]
\subsection{Prueba 3 : tiempo de Operación del producto punto}
% \kant[1-2]
\begin{table}[htb]
\caption{Algoritmo de prueba con 2GB de carga}
\label{1234}
\resizebox{\columnwidth}{!}{%
\begin{tabular}{c*{5}{>{$}c<{$}}}
& \text{Núcleos(GB)}  & \text{MPI (segs.)} & \text{BigMPI (segs.)}\\
& 2 & 10.1 & 10.1\\
& 4  & 6.88 & 6.87\\
& 6  & 5.74 & 5.76\\
& 8    & 5.34 & 5.48\\
& 10    & 5.05 & 5.06\\
& 12    & 4.86 & 4.86
\end{tabular}
}
\end{table}
% \kant[3-5]

\section{Conclusiones}
En el presente trabajo se encontró el inconveniente del límite de memoria disponible
en nuestras computadoras de laboratorio (8GB de RAM), que durante nuestras pruebas
ocasionaron que la computadora master se cuelgue con counts más allá de $2^{31}$.

Se realizaron las pruebas de rendimiento utilizando un cluster
de computadoras y en el que se concluyó el poco overhead añadido considerando los pocos
datos con los que logramos probar (debido al problema descrito en el parrafo anterior).

Se verificó el funcionamiento adecuado de las funciones extendidas de BigMPI,
tales como: \texttt{MPIX\_Send\_x}
y \texttt{MPIX\_Recv\_x} que nos permiten transmitir más allá de los $2^{31}$ elementos
que las que nos permiten realizar las funciones estándar de MPI,

Se evidencia la importancia de este aporte, tanto que ya se considera su soporte en MPI de forma
nativa para la especificación Nro. 4 que aún no ha salido~\cite{mpi4}.
\bibliographystyle{IEEEtran}
\bibliography{IEEEabrv,bib/biblio,bib/petsc}
\onecolumn
\appendices
\section{Código fuente usado en la medición}
\lstinputlisting[caption=Medición del ancho de banda con MPI, style=customc]{code/bandwidth_mpi.c}
\lstinputlisting[caption=Medición del ancho de banda con BigMPI, style=customc]{code/bandwidth_bigmpi.c}
\lstinputlisting[caption=prueba de envío y recepción con mpi, style=customc]{code/test_send_recv_mpi.c}
\lstinputlisting[caption=prueba de envío y recepción con bigmpi, style=customc]{code/test_send_recv_bigmpi.c}
\lstinputlisting[caption=prueba de producto punto con BigMPI, style=customc]{code/prueba_x_bigmpi.c}
\lstinputlisting[caption=prueba de producto punto con MPI, style=customc]{code/prueba_x_mpi.c}
\clearpage
\newpage
\twocolumn

% needed in second column of first page if using \IEEEpubid
%\IEEEpubidadjcol

% An example of a floating figure using the graphicx package.
% Note that \label must occur AFTER (or within) \caption.
% For figures, \caption should occur after the \includegraphics.
% Note that IEEEtran v1.7 and later has special internal code that
% is designed to preserve the operation of \label within \caption
% even when the captionsoff option is in effect. However, because
% of issues like this, it may be the safest practice to put all your
% \label just after \caption rather than within \caption{}.
%
% Reminder: the "draftcls" or "draftclsnofoot", not "draft", class
% option should be used if it is desired that the figures are to be
% displayed while in draft mode.
%
%\begin{figure}[!t]
%\centering
%\includegraphics[width=2.5in]{myfigure}
% where an .eps filename suffix will be assumed under latex,
% and a .pdf suffix will be assumed for pdflatex; or what has been declared
% via \DeclareGraphicsExtensions.
%\caption{Simulation Results}
%\label{fig_sim}
%\end{figure}

% Note that IEEE typically puts floats only at the top, even when this
% results in a large percentage of a column being occupied by floats.


% An example of a double column floating figure using two subfigures.
% (The subfig.sty package must be loaded for this to work.)
% The subfigure \label commands are set within each subfloat command, the
% \label for the overall figure must come after \caption.
% \hfil must be used as a separator to get equal spacing.
% The subfigure.sty package works much the same way, except \subfigure is
% used instead of \subfloat.
%
%\begin{figure*}[!t]
%\centerline{\subfloat[Case I]\includegraphics[width=2.5in]{subfigcase1}%
%\label{fig_first_case}}
%\hfil
%\subfloat[Case II]{\includegraphics[width=2.5in]{subfigcase2}%
%\label{fig_second_case}}}
%\caption{Simulation results}
%\label{fig_sim}
%\end{figure*}
%
% Note that often IEEE papers with subfigures do not employ subfigure
% captions (using the optional argument to \subfloat), but instead will
% reference/describe all of them (a), (b), etc., within the main caption.


% An example of a floating table. Note that, for IEEE style tables, the
% \caption command should come BEFORE the table. Table text will default to
% \footnotesize as IEEE normally uses this smaller font for tables.
% The \label must come after \caption as always.
%
%\begin{table}[!t]
%% increase table row spacing, adjust to taste
%\renewcommand{\arraystretch}{1.3}
% if using array.sty, it might be a good idea to tweak the value of
% \extrarowheight as needed to properly center the text within the cells
%\caption{An Example of a Table}
%\label{table_example}
%\centering
%% Some packages, such as MDW tools, offer better commands for making tables
%% than the plain LaTeX2e tabular which is used here.
%\begin{tabular}{|c||c|}
%\hline
%One & Two\\
%\hline
%Three & Four\\
%\hline
%\end{tabular}
%\end{table}


% Note that IEEE does not put floats in the very first column - or typically
% anywhere on the first page for that matter. Also, in-text middle ("here")
% positioning is not used. Most IEEE journals use top floats exclusively.
% Note that, LaTeX2e, unlike IEEE journals, places footnotes above bottom
% floats. This can be corrected via the \fnbelowfloat command of the
% stfloats package.



% if have a single appendix:
%\appendix[Proof of the Zonklar Equations]
% or
%\appendix  % for no appendix heading
% do not use \section anymore after \appendix, only \section*
% is possibly needed

% use appendices with more than one appendix
% then use \section to start each appendix
% you must declare a \section before using any
% \subsection or using \label (\appendices by itself
% starts a section numbered zero.)
%

% begin dennisbot
% \appendices
% \section{Proof of the First Zonklar Equation}
% \blindtext

% use section* for acknowledgement
% \section*{Acknowledgment}

% end dennisbot

% The authors would like to thank...


% Can use something like this to put references on a page
% by themselves when using endfloat and the captionsoff option.
\ifCLASSOPTIONcaptionsoff
  \newpage
\fi



% trigger a \newpage just before the given reference
% number - used to balance the columns on the last page
% adjust value as needed - may need to be readjusted if
% the document is modified later
%\IEEEtriggeratref{8}
% The "triggered" command can be changed if desired:
%\IEEEtriggercmd{\enlargethispage{-5in}}

% references section

% can use a bibliography generated by BibTeX as a .bbl file
% BibTeX documentation can be easily obtained at:
% http://www.ctan.org/tex-archive/biblio/bibtex/contrib/doc/
% The IEEEtran BibTeX style support page is at:
% http://www.michaelshell.org/tex/ieeetran/bibtex/
%\bibliographystyle{IEEEtran}
% argument is your BibTeX string definitions and bibliography database(s)
%\bibliography{IEEEabrv,../bib/paper}
%
% <OR> manually copy in the resultant .bbl file
% set second argument of \begin to the number of references
% (used to reserve space for the reference number labels box)

% \begin{thebibliography}{1}

% \bibitem{IEEEhowto:kopka}
% H.~Kopka and P.~W. Daly, \emph{A Guide to \LaTeX}, 3rd~ed.\hskip 1em plus
%   0.5em minus 0.4em\relax Harlow, England: Addison-Wesley, 1999.

% \end{thebibliography}

% \bibliographystyle{IEEEtran}
% \bibliography{IEEEabrv,bib/biblio,bib/petsc}

% biography section
%
% If you have an EPS/PDF photo (graphicx package needed) extra braces are
% needed around the contents of the optional argument to biography to prevent
% the LaTeX parser from getting confused when it sees the complicated
% \includegraphics command within an optional argument. (You could create
% your own custom macro containing the \includegraphics command to make things
% simpler here.)
%\begin{biography}[{\includegraphics[width=1in,height=1.25in,clip,keepaspectratio]{mshell}}]{Michael Shell}
% or if you just want to reserve a space for a photo:

% \begin{IEEEbiography}[{\includegraphics[width=1in,height=1.25in,clip,keepaspectratio]{picture}}]{John Doe}
% \blindtext
% \end{IEEEbiography}

% You can push biographies down or up by placing
% a \vfill before or after them. The appropriate
% use of \vfill depends on what kind of text is
% on the last page and whether or not the columns
% are being equalized.

%\vfill

% Can be used to pull up biographies so that the bottom of the last one
% is flush with the other column.
%\enlargethispage{-5in}




% that's all folks
\end{document}


